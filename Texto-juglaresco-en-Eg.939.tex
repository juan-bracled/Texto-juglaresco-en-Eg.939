\documentclass[11pt,a4paper,twoside]{article}
\usepackage[noeledsec,nocritical,noend,noledgroup,nopenalties,series={E}]{reledmac}
\usepackage{fontspec}
\usepackage[paperwidth=17.00cm,paperheight=25.00cm,top=1.40cm,inner=1.95cm,outer=1.65cm,bottom=2.54cm]{geometry}
\usepackage{polyglossia}
\usepackage[backend=biber,style=nncph]{biblatex}
\addbibresource{bibliography.bib}
\usepackage[breaklinks=true,hidelinks]{hyperref}
\usepackage{graphicx}
%
\parindentX
\setlength{\parindent}{0.6cm}
\setlength{\bibhang}{0.6cm}
\AtBeginDocument{\maxhnotesX{1\textheight}}
\linenummargin{outer}
%
\newcommand{\suppl}[1]{〈#1〉}
\setdefaultlanguage{spanish}
\setmainfont{TemporaNR}[RawFeature=lnum]
%
\usepackage{enumitem}
\setlist{nosep,noitemsep,parsep=0pt}
%
%
\begin{document}
%
Texto extraído del manuscrito \emph{Cancionero de Egerton} (British Library, Eg. 939), ff. 25v-26r. <\url{https://www.cervantesvirtual.com/nd/ark:/59851/bmcj1034}>.\par
%
Criterios de transcripción:
%
\begin{itemize}[label=--]%
\item Se respetan las grafías del manuscrito, salvo \emph{rr} y la alternancia \emph{v}--\emph{u}.\par
\item Las abreviaturas se desarrollan en cursiva.\par
\item La nomenclatura sacra χ{\char"00305}ρ{\char"00305}ο[s] se desarrolla como \emph{Christo} o \emph{christianos}.\par
\item Las grafías entre [] en el texto indican que hay una pequeña laguna por alguna mancha o daño en el papel.\par 
\item Se añaden numeraciones entre [] a cada estrofa para separarlas.\par
\item \emph{pler. suppl.} \suppl{} \textcite{Severin2000}.\par
\end{itemize}\par
\vskip6pt
%
Criterios de modernización:
%
\begin{itemize}[label=--]%
\item Por la sibilante \emph{ç}, cuando va delante de \emph{i} y \emph{e}, se coloca \emph{c}.\par
\item Por \emph{q} en las palabras de raíz \emph{qua}- se coloca \emph{c}.\par
\item Por \emph{y}, cuando no es una conjunción copulativa, se coloca \emph{i}.\par
\item Se suprime la alternancia \emph{v}--\emph{u}.\par
\item Las \emph{n} en las siglas como «nin» se quitan. En los vocablos cuyo uso es el equivalente actual de /\emph{m}/, como en \emph{tiempo}, se coloca \emph{m}.\par
\item Por \emph{t}, al finalizar un vocablo, se pone \emph{d}, y se quita \emph{d} en «grand».\par
\item Los vocablos en cursiva y con una explicación indican que se cambió el vocablo utilizado en el manuscrito por otro equivalente --o próximo-- y accesible.\par
\end{itemize}
%
\section*{\textsc{Bibliografía}}
  \nocite{*}
  \printbibliography[heading=none]
\newpage
%
\subsection*{\raggedleft\textsc{Agora es tienpo de ganar buena soldada}}
%
\noindent[f. 25v] ¶Para los devotos christianos que está\emph{n} en la batalla espiritual aplícanse estos metros sobre el ca\emph{n}tar que dize\emph{n} los juglares: «agora es tie\emph{n}po de ganar buena soldada».\par%
%
\beginnumbering
\begin{center}
\textbf{[I]}
\end{center}
\pstart
¶Pues tienes libre poder\\
de pelear y vençer,\\
date priesa a meresçer\\
la perdurable morada,\par
¶agora es tie\emph{n}po de ganar buena soldada.\footnoteE{El título de la trova es un estribillo. Probablemente cuando los juglares habían de recitar en público, estos repetían el estribillo tras cada estrofa, porque la rima cuarta de cada una acaba con -\emph{ada}, que perfectamente rima con \emph{soldada}. Es plausible considerar que el amanuense, por diferentes razones, omitió la repetición del estribillo en el texto escrito, quizás asumiendo que su destinatario conocía la composición, o no consideró necesaria la redacción del texto del estribillo.}\par
\pend
%
\begin{center}
\textbf{[II]}
\end{center}
\pstart
¶Que después del hombre muerto\\
es çierto y mucho çierto\\
que de q\emph{ua}nto fizo tuerto\\
le será cue\emph{n}ta tomada.\par
\pend
%
\begin{center}
\textbf{[III]}
\end{center}
\pstart
¶Has de yr solo, pelegrino,\\
tras los pasos del dios trino,\\
mirándote tan indino\\
que te hizo de no nada,\par
¶agora es tie\emph{n}po...\par
\pend
%
\begin{center}
\textbf{[IV]}
\end{center}
\pstart
¶Sirvie\emph{n}do prude\emph{n}teme\emph{n}te\\
simple, ma\emph{n}so y dilige\emph{n}te,\\
creyendo que está prese\emph{n}te\\
el que es fin de la jornada.\par
\pend
%
\begin{center}
\textbf{[V]}
\end{center}
\pstart
¶Por sus obras has de andar,\\
en çierto sin vagar,\\
y si\emph{n} dezir ni pensar\\
palabra demasiada.\par
\pend
%
\begin{center}
\textbf{[VI]}
\end{center}
\pstart
¶Y los siervos bien midridos\\
trayg[an s]ienpre los sentidos,\\
reçe[len] los recogidos\\
del temor de la çelada,\par
¶agora es tienpo de ganar buena soldada.\par
\pend
%
\begin{center}
\textbf{[VII]}
\end{center}
\pstart
\noindent[f. 26r] ¶Por este mu\suppl{n}do mezq\emph{ui}no\\
pasa apriesa y de camino,\\
q\emph{ua}nto tomes pa\emph{n} y vino\\
no estés más en la posada.\par
\pend
%
\begin{center}
\textbf{[VIII]}
\end{center}
\pstart
¶Sey continuo en la oraçió\emph{n}\\
co\emph{n} fervor de co\emph{n}triçió\emph{n}\footnoteE{Esto significa «orar con arrepentimiento».},\\
siempre puesta el ate\emph{n}çión\\
en la verdat encarnada.\par
\pend
%
\begin{center}
\textbf{[IX]}
\end{center}
\pstart
¶Busca secretos lugares\\
do pienses en lo q\emph{ue} errares,\\
y gozes de los lugares\\
de la culpa bie\emph{n} llorada.\par
\pend
%
\begin{center}
\textbf{[X]}
\end{center}
\pstart
¶Lo que te sobra del día\\
gástalo co\emph{n} quien te guía,\\
tratando del alegría\\
de la gl\emph{or}ia deseada.\par
\pend
%
\begin{center}
\textbf{[XI]}
\end{center}
\pstart
¶Y vendrás a la vmildat\\
que es creer de ti v\emph{er}dat,\\
de la santa caridat\\
suele estar ap\suppl{o}sentada.\par
\pend
\relax
\vfill
\newpage
%
\begin{center}
\textbf{[XII]}
\end{center}
\pstart
¶Y gusta\emph{n}do los dulçores\\
de sus muy altos amores,\\
a los tres co\emph{n}petidores\\
ya no los ternas en nada.\par
\pend
%
\begin{center}
\textbf{[XIII]}
\end{center}
\pstart
¶Do vernas a co\emph{n}te\emph{n}plar\\
n\emph{uest}ra grand gl\emph{or}ia sin par\\
y del todo a despreçiar\\
esta tu carne cuytada.\par
\pend
%
\begin{center}
\textbf{[XIV]}
\end{center}
\pstart
¶Si tovieres el q\emph{ue}rer\\
todo lo puedes aver,\\
que esto venimos fazer\\
en esta tierra prestada.\par
\pend
%
\begin{center}
\textbf{[XV]}
\end{center}
\pstart
¶Lo q\emph{ue} Adán nos perdió,\\
Ih\emph{e}su Chr\emph{is}to\footnoteE{Emula la nomina sacra corriente durante el medioevo, cuya utilización se atestigua en los antiguos
manuscritos unciales griegos de los textos neotestamentarios, tales como el \emph{Papiro 1}, el \emph{Codex Sinaticus} o el \emph{Codex Bezae Cantabrigiensis}, entre otros. {ις{\char"00305}} {χς{\char"00305}} = Ιησους Χριστος.} lo cobró\\
co\emph{n} su muerte y nos ganó\\
vida bie\emph{n} ave\emph{n}turada.\par
\pend
%
\begin{center}
\textbf{[XVI]}\par
¶\textbf{Cabo}
\end{center}
\pstart
¶Pues yo ruego al q\emph{ue} esto l\suppl{ea}\\
q\emph{ue} lo obre y q\emph{ue} me crea,\\
por q\emph{ue} goze desta prea\\
q\emph{ue} por dios nos fue ganada.\footnoteE[4]{¿Es plausible considerar que el cabo o estrofa de cierre sea un añadido del copista por cuenta propia o por orden ajena? Probablemente lo sea porque la rúbrica anuncia que es un cantar \emph{que dizen los juglares} y, por tanto, pertenecería a la tradición oral y no escrita y, de ser la \emph{enmendatio} «l\suppl{ea}» --realizada por \textcite{Severin2000}-- la lección correcta, la estrofa que cierra el texto estaría falseando la propia oralidad intrínseca a la tradición juglaresca, al enviar o invitar a leer y no a oír o «atender» lo que se recita.}\par
\pend
\endnumbering
\relax
\vfill
\newpage
%
\subsection*{\raggedleft\textsc{Ahora es tiempo de ganar buena soldada}}
%
Para los devotos cristianos que están en la batalla espiritual se aplican estos metros de la canción que llaman los juglares: «\emph{ahora es tiempo de ganar buena soldada}»\footnoteE{«soldada» = de sueldo, recibir un pago por las acciones o un trabajo. El texto apunta a que el «pago» a recibir es el «reino divino», prometido a quienes sigan al Cristo.}.\par
%
\beginnumbering
\begin{center}
\textbf{[I]}
\end{center}
\pstart
\textsc{Pues tienes} libre poder\\
de pelear y vencer,\\
date prisa a merecer\\
la perdurable morada,\par
\emph{ahora es tiempo de ganar buena soldada}.\par
\pend
%
\begin{center}
\textbf{[II]}
\end{center}
\pstart
Que después del hombre muerto\\
es cierto y mucho cierto\\
que de cuanto hizo tuerto\\
le será cuenta tomada.\par
\pend
%
\begin{center}
\textbf{[III]}
\end{center}
\pstart
Has de ir solo, peregrino,\\
tras los pasos del Dios trino,\\
mirándote tan indigno\\
que te hizo de no nada,\par
\emph{ahora es tiempo}...\par
\pend
%
\begin{center}
\textbf{[IV]}
\end{center}
\pstart
Sirviendo prudentemente,\\
simple, manso y diligente,\\
creyendo que está presente\\
el que es fin de la jornada.\par
\pend
%
\begin{center}
\textbf{[V]}
\end{center}
\pstart
Por sus obras has de andar,\\
en cierto sin vagar\\
y sin decir ni pensar\\
palabra demasiada.\par
\pend
\relax
\vfill
\newpage
%
\begin{center}
\textbf{[VI]}
\end{center}
\pstart
Y los siervos bien \emph{abiertos}\footnoteE{\emph{Y los siervos bien midridos} / \emph{traygan sienpre los sentidos} = el vocablo adjetivo «midridos» parece provenir del vocablo griego \emph{μυδρίασις}, que significa «dilatación de la pupila» [\href{https://logeion.uchicago.edu/\%CE\%BC\%CF\%85\%CE\%B4\%CF\%81\%CE\%AF\%CE\%B1\%CF\%83\%CE\%B9\%CF\%82}{https://logeion.uchicago.edu/μυδρίασις}]; de ser así la etimología para el adjetivo utilizado, el sentido sería, entonces, que los siervos deberían traer «los sentidos bien abiertos» o «estar muy atentos».}\\
traigan siempre los sentidos,\\
recelen los recogidos\\
del temor de la celada,\par
\emph{ahora es tiempo de ganar buena soldada}.\par
\pend
%
\begin{center}
\textbf{[VII]}
\end{center}
\pstart
Por este mundo mezquino\\
pasa aprisa y de camino,\\
cuanto tomes pan y vino\\
no estés más en la posada.\par
\pend
%
\begin{center}
\textbf{[VIII]}
\end{center}
\pstart
Sé continuo en la oración\\
con fervor de contrición,\\
siempre puesta la atención\\
en la verdad encarnada.\par
\pend
%
\begin{center}
\textbf{[IX]}
\end{center}
\pstart
Busca secretos lugares\\
do pienses en lo que errares,\\
y goces de los lugares\\
de la culpa bien llorada.\par
\pend
%
\begin{center}
\textbf{[X]}
\end{center}
\pstart
Lo que te sobra del día\\
gástalo con quien te guía,\\
tratando de la alegría\\
de la gloria deseada.\par
\pend
%
\begin{center}
\textbf{[XI]}
\end{center}
\pstart
Y vendrás a la humildad\\
que es creer de ti verdad,\\
de la santa caridad\\
suele estar aposentada.\par
\pend
%
\begin{center}
\textbf{[XII]}
\end{center}
\pstart
Y gustando los dulzores\\
de sus muy altos amores,\\
a los tres competidores\\
ya no los \emph{tendrás}\footnoteE{«ternas» en el \emph{cod.} Proviene de «ternir», un arcaísmo castellano que significa «tener». En este caso es «tendrás» (\emph{seg. sg. fut.}). Lo atestigua el manuscrito \textit{M} del \emph{Libro de los doze sabios}, ubicado en la Biblioteca Menéndez Pelayo de Santander. <\url{https://parnaseo.uv.es/Memorabilia/Memorabilia6/listillos/menu.htm}>. Para la significación \emph{cfr.} <\url{https://historicas.unam.mx/publicaciones/publicadigital/monarquia/volumen/07/miv7042.pdf}>} en nada.\par
\pend
%
\begin{center}
\textbf{[XIII]}
\end{center}
\pstart
Do \emph{vendrás}\footnoteE{«vernas» en el \emph{cod.} Es un arcaísmo castellano de «venir», que significa «vendrás» (\emph{seg. sg. fut.}). Atestigua este vocablo el manuscrito de la \emph{Fazienda de Ultramar} (Univ. de Salamanca), año 1220-1230.
<\url{http://purl.unirostock.de/demel/k0355}>} a contemplar\\
nuestra gran gloria sin par\\
y del todo a despreciar\\
esta tu carne cuitada.\par
\pend
%
\begin{center}
\textbf{[XIV]}
\end{center}
\pstart
Si tuvieras el querer\\
todo lo puedes \emph{tener}\footnoteE{«\emph{Si touieres el querer} / \emph{todo lo puedes aver}» en el \emph{cod.} = quizá el sentido de esta estrofa apuntaría a que con el querer [a Cristo --por contexto--] se llegará a la plenitud de «tenerlo todo», según lo indica el arcaísmo \emph{aver}, que proviene del latín \emph{habēre} [tener, poseer]; a eso, el \emph{DCECH} indica que ya en el \emph{Mio Cid} el vocablo \emph{aver} había iniciado a ser desplazado por \emph{tener} en el s. XII; y del \emph{Cid} se cita «lanças que todas \emph{tienen} pendones».},\\
que esto venimos \suppl{a} hacer\\
en esta tierra prestada.\par
\pend
%
\begin{center}
\textbf{[XV]}
\end{center}
\pstart
Lo que Adán nos perdió,\\
Jesu-Cristo lo cobró\\
con su muerte y nos ganó\\
vida bienaventurada.\par
\pend
%
\begin{center}
\textbf{[XVI]}\par
\textbf{Cabo}
\end{center}
\pstart
Pues yo ruego al que esto lea,\\
que lo obre y que me crea,\\
para que goce de esta prea\\
que por Dios nos fue ganada.\par
\pend
\endnumbering
\relax
\vfill
\newpage
%
\section*{\normalsize \textsc{Imágenes del códice}}
%
{\centering
\includegraphics[width=1\textwidth]{25v.png}}
Fol. 25v = tras acabar el texto del \emph{Infante Epitus}, inicia el texto juglaresco.\par
\newpage
%
{\centering
\includegraphics[width=1\textwidth]{26r.png}}
%
\end{document}